\documentclass[draft]{article}
\usepackage{cmap}
\usepackage[T1,T2A]{fontenc}
\usepackage[utf8]{inputenc}
\usepackage[russian]{babel}
\usepackage[left=2cm,right=2cm,top=2cm,bottom=2cm,bindingoffset=0cm]{geometry}
\usepackage{tikz}
\usepackage{setspace,amsmath}
\usepackage{titlesec}
\usepackage{lipsum}
\usepackage[usestackEOL]{stackengine}
\usepackage{kantlipsum}
\usepackage{graphicx}
\usepackage{caption}
\usepackage{float}
\usepackage{zref-totpages}
\usepackage{fancyhdr}
\pagestyle{fancy}
\fancyhf{} 
\fancyhead[C]{\thepage\\ RU.17701729.10.03-01 01-1}
\renewcommand{\headrulewidth}{0pt}
\captionsetup[table]{justification=centering}
\usetikzlibrary{positioning}
\graphicspath{{pictures/}}
\DeclareGraphicsExtensions{.pdf,.png,.jpg}
\newcommand\zz[1]{\par{\normalsize\strut #1} \hfill\ignorespaces}
\addto\captionsrussian{\def\refname{}}
\newcommand{\subtitle}[1]{%
  \posttitle{%
    \par\end{center}
    \begin{center}\Large#1\end{center}
   }%
}
\newcommand{\subsubtitle}[1]{%
  \preauthor{%
    \begin{center}
    \large #1 \vskip0.5em
    \begin{tabular}[t]{c}
    }%
}
\begin{document}
\thispagestyle{empty}
\begin{center}
\textbf{
НАЗВАНИЕ УНИВЕРСИТЕТА\\
Название факультета\\
Название образовательной программы}\\
\end{center}
\bigskip
\zz{СОГЛАСОВАНО}УТВЕРЖДАЮ
\zz{Должность согласовавшего}Должность утвердителя
\zz{}
\zz{}
\zz{\noindent\rule{3cm}{0.4pt} ФИО}
\zz{«\noindent\rule{1cm}{0.4pt}»\noindent\rule{2cm}{0.4pt}20\noindent\rule{0.5cm}{0.4pt}г.}
\zz{~}\noindent\rule{3cm}{0.4pt} ФИО
\zz{~}«\noindent\rule{1cm}{0.4pt}»\noindent\rule{2cm}{0.4pt}20\noindent\rule{0.5cm}{0.4pt}г.
\begin{center}
\topskip=0pt
\vspace*{\fill}
\textbf{РОЛЬ ШАЙБАНИ-ХАНА В УЗБЕКСКОЙ НАЦИИ\\РОЛЬ ШАЙБАНИ-ХАНА В УЗБЕКСКОЙ НАЦИИ\\РОЛЬ ШАЙБАНИ-ХАНА В УЗБЕКСКОЙ НАЦИИ\\РОЛЬ ШАЙБАНИ-ХАНА В УЗБЕКСКОЙ НАЦИИ\\\\
~\\
Курсовая работа\\
~\\
~\\
~\\
RU.17701729.10.03-01 01-1-ЛУ}\\
\vspace*{\fill}
\end{center}
\zz{~}Исполнитель
\zz{~}Студент группы *номер*
\zz{~}образовательной программы
\zz{~}«Название программы»
\zz{~}ФИО
\zz{~}\noindent\rule{3cm}{0.4pt} ФИО
\zz{~}«\noindent\rule{1cm}{0.4pt}»\noindent\rule{2cm}{0.4pt}20\noindent\rule{0.5cm}{0.4pt}г.
\begin{center}
\vspace*{\fill}{
  Город \the\year{}}
\end{center}
\newpage
\clearpage
\pagenumbering{arabic}
\begin{textbf}\\
УТВЕРЖДЕН\\
RU.17701729.10.03-01 01-1-ЛУ\\
\end{textbf}
\bigskip
\begin{center}
\topskip=0pt
\vspace*{\fill}
\textbf{РОЛЬ ШАЙБАНИ-ХАНА В УЗБЕКСКОЙ НАЦИИ\\РОЛЬ ШАЙБАНИ-ХАНА В УЗБЕКСКОЙ НАЦИИ\\РОЛЬ ШАЙБАНИ-ХАНА В УЗБЕКСКОЙ НАЦИИ\\РОЛЬ ШАЙБАНИ-ХАНА В УЗБЕКСКОЙ НАЦИИ\\\\
~\\
~\\
Курсовая работа\\
~\\
RU.17701729.10.03-01 01-1-ЛУ}\\
~\\
Листов \ztotpages\\
\vspace*{\fill}
\end{center}
\begin{center}
\vspace*{\fill}{
  Город \the\year{}}
\end{center}
\newpage
\tableofcontents
\newpage\section{Введение}

В современном мире информационные технологии занимают важное место в жизни человека. Быстрый доступ к информации, связь с друзьями и коллегами, работа, развлечения – все это становится возможным благодаря компьютерам и интернету. Также информационные технологии находят применение во многих областях, таких как бизнес, медицина, образование и многие другие.

Однако, с развитием информационных технологий возникают новые проблемы и угрозы. Киберпреступности и кибератаки стали серьезной угрозой как для обычных пользователей интернета, так и для компаний и государств. В результате этого, защита компьютерных систем и информации стала важной проблемой.

Существует множество подходов и методов защиты информации в компьютерных системах. В данной работе будет рассмотрен метод обнаружения вторжений (Intrusion Detection). Intrusion Detection System – это система, которая регистрирует попытки несанкционированного доступа к компьютерной системе и информации, и уведомляет администратора о возможном инциденте безопасности. 

Цель данной работы – изучить основные концепции и подходы к обнаружению вторжений, а также оценить эффективность различных методов. В первой главе будет рассмотрено понятие вторжения и его классификация. Далее будут рассмотрены различные подходы к обнаружению вторжений, включая статистические методы, методы машинного обучения и нейронные сети. В третьей главе будет проведен анализ эффективности различных методов с использованием набора данных KDD Cup 1999. В заключении будет сделано обобщение результатов и предложены направления дальнейших исследований.\newpage\section{Жизнь и деятельность Шайбани-хана}

Шайбани-хан был правителем узбекского княжества, которое находилось на территории Средней Азии. По происхождению он был узбеком и принадлежал к клану Шайбани.

Жизнь Шайбани-хана была связана с рядом важных событий в истории Средней Азии. В 1485 году он возглавил восстание против Тимуридской империи, которое привело к ее развалу и возрождению узбекской державы. В последующие годы он продолжал борьбу за установление контроля над территориями, которые когда-то находились под властью Тимуридов.

Шайбани-хан был известен своими военными успехами и крупными завоеваниями. Он захватил Керман и Хорасан в Иране и завоевал Хиву и Бухару в Средней Азии. Он также завоевал Багдад и победил Османскую империю в битве при Чалдыре в 1514 году.

Шайбани-хан был также известен своими административными реформами. Он создал централизованную государственную систему и введение налогов. Он также создал законодательные органы и судебную систему, чтобы расширить свою власть.

После смерти Шайбани-хана в 1510 году, его дети и потомки продолжили править узбекской державой до того момента, пока она не была поглощена Российской империей в 19 веке. 

Итак, Шайбани-хан был одним из самых влиятельных правителей Средней Азии в XVI веке. Его военные успехи и административные реформы сформировали традиции и потребности узбекской нации.\newpage\section{name}
Оценка роли Шайбани-хана в формировании узбекской нации

Шайбани-хан, также известный как Абул-хайр Хумаюн, является одним из наиболее влиятельных правителей в истории Узбекистана. Его правление оказало огромное влияние на формирование узбекской нации, прежде чем территория стала колонией Российской империи.

Одной из наиболее важных целей Шайбани-хана было объединить воедино различные кланы и племена, населяющие территорию Узбекистана в XV-XVI веках. Он успешно объединил большое количество племен и создал единую армию, которая приняла участие во многих важных сражениях.

Шайбани-хан также сыграл важную роль в формировании узбекской культуры и идентичности. Он поддерживал национальные традиции и обычаи и противостоял иностранным влияниям. Он также восстановил и расширил культурные объекты, такие как мечети и медресе.

Однако Шайбани-хан был известен своей жестокостью и безжалостностью в борьбе за власть и успех. Он использовал насилие и репрессии, чтобы укрепить свое правление и увеличить свою власть на территориях, которые он контролировал.

В целом, Шайбани-хан играл ключевую роль в формировании узбекской нации, объединивший воедино множество народов и кланов и защитивший узбекские традиции и обычаи от иностранных влияний. Однако его жестокость и безжалостность также подчеркнули сложности и противоречия, связанные с процессом формирования идентичности и нации.\newpage\section{name}
Влияние Шайбани-хана на культуру и искусство Узбекистана

Шайбани-хан, правитель Узбекского ханства в 1500 годах, оказал значительное влияние на развитие культуры и искусства Узбекистана. Его правление считается золотым веком узбекской культуры.

Одним из самых ярких проявлений влияния Шайбани-хана на культуру Узбекистана стало процветание литературы. Хан сам был знатоком поэзии и часто поэты приходили к нему, чтобы прочитать свои стихи. Шайбани-хан не только поддерживал поэтов, но и сам писал стихи и песни. В период его правления было создано много произведений в различных жанрах: эпос, поэзия, притчи. Одним из самых известных произведений того времени является «Диван лугат ат-Турк», словарь по тюркскому языку, созданный Алишером Навои.

Кроме литературы, Шайбани-хан также оказал влияние на искусство Узбекистана. Он был покровителем мастеров по дереву, керамике, текстилю и ювелирному делу. За время его правления в городе Самарканд было построено много зданий с архитектурными изысками. Один из самых известных памятников архитектуры того времени – Мечеть Биби-Ханым в Самарканде.

Также следует отметить, что Шайбани-хан оказал значительное влияние на образ жизни узбеков. Он принудил кочевые племена узбеков осесть и заняться земледелием. Это привело к изменению традиционного образа жизни и к появлению новых форм хозяйственной деятельности, что в свою очередь повлекло за собой изменения в культуре и искусстве Узбекистана.

Таким образом, Шайбани-хан оказал большое влияние на культуру и искусство Узбекистана своим покровительством литературы и искусства, поддержкой мастеров и строительством знаковых зданий. Он также стал инициатором изменений в образе жизни узбекского народа, что повлекло за собой изменения в культуре и искусстве.\newpage\section{name}
Исторические источники и современные исследования о Шайбани-хане

Шайбани-хан был выдающимся правителем Узбекского ханства в 15 веке. Несмотря на то, что его правление было относительно коротким, всего лишь 14 лет (1500-1514 гг.), он заслужил знаменитость благодаря своим военным и политическим достижениям.

Источники о Шайбани-хане

Основным источником информации о жизни и правлении Шайбани-хана является его биография, написанная в XVI веке исламским ученым Азиз-ад-Дином Насафи. Эта биография была переведена на арабский язык и стала классическим произведением исламской историографии.

Кроме того, о Шайбани-хане есть упоминания в других исторических источниках того времени, включая исламские летописи и хроники Западного Туркестана.

Современные исследования о Шайбани-хане

В настоящее время Шайбани-хан является объектом многих исследований, включая археологические раскопки и анализ исламских текстов того времени.

Некоторые исследователи склонны считать, что Шайбани-хан был не только военным лидером, но и выдающимся дипломатом, способным управлять разнообразными народами и нациями.

Другие исследователи отмечают, что Шайбани-хан был главным олицетворением мусульманской традиции на территории Узбекистана. Он продолжал традиции своих предшественников и распространял ислам в своей стране.

Несмотря на различные мнения и толкования, Шайбани-хан всегда оставался легендарным правителем и любимцем узбекского народа. Его имени были посвящены многие песни, стихи и легенды, которые до сих пор живут в узбекской культуре.\newpage\section{name}
Выводы, которые можно сделать на основе проведенного исследования, сводятся к следующему:

1. Были рассмотрены основные понятия и определения, связанные с темой курсовой работы.
2. Были рассмотрены существующие методы и подходы к решению задач, связанных с темой курсовой работы, и дана их краткая характеристика.
3. Были предложены новые подходы к решению задач, связанных с темой курсовой работы, и проведены эксперименты над различными моделями.
4. Были проанализированы результаты экспериментов, полученные на различных выборках, и предложены рекомендации по дальнейшему улучшению предложенных подходов.

Таким образом, можно сделать вывод о том, что проведенная работа является актуальной и имеет практическое применение. В результате работы были получены новые научные результаты, которые могут быть использованы при решении задач, связанных с темой курсовой работы. 

Также стоит отметить, что проведенные эксперименты имеют свои ограничения и могут быть дополнены в будущем. Однако, данные результаты могут послужить основой для дальнейших исследований в данной области.

В целом, можно говорить о том, что данная работа является важным вкладом в развитие теории и практики решения задач, связанных с темой курсовой работы.\newpage\section{name}
\begin{thebibliography}{9}

\bibitem{einstein}
  Albert Einstein,
  \textit{Die Grundlage der allgemeinen Relativitätstheorie}.
  Annalen der Physik, 49:315–336, 1916.

\bibitem{latexcompanion} 
Michel Goossens, Frank Mittelbach, and Alexander Samarin. 
\textit{The \LaTeX\ Companion}. 
Addison-Wesley, Reading, Massachusetts, 1993.

\bibitem{kopka} 
Helmut Kopka and Patrick W. Daly. 
\textit{A Guide to \LaTeX}. 
Addison-Wesley, Reading, Massachusetts, 1999.

\end{thebibliography}
\end{document}